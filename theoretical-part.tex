\section{Матрица чувствительности}
Для удобного рассмотрения модели введем обозначения:

$ V_P $ -- множество узлов с заданным давлением,
$ V_Q $ -- множество узлов с заданным притоком

Рассмотрим случай, когда замыкающие соотношения являются непрерывно диффиренцируемыми в окрестности
решения $ P_0, Q_0 $ системы (5).

Обозначим 

$$ d_{Fi} = \frac{\partial \phi_i(P_S, P_F) }{\partial P_S} $$

$$ d_{Li} = -\frac{\partial \phi_i(P_S, P_F) }{\partial P_F} $$

Тогда в силу монотонности $ \phi_i $ справедливы неравенства $ d_{Fi} \geq 0 $ и $ d_{Li} \geq 0 $

Определим диагональные матрицы $ D_F $ и $ D_L $ с $ d_{Fi} $ и $ d_{Li} $ на диагонали.
Тогда уравнения (2) и (3) можно переписать 

$$ dX = (D_F A_S^T + D_L A_F^T) dP \eqno(7) $$
$$ dQ = A (D_F A_S^T + D_L A_F^T) dP \eqno(8) $$

Перенумеруем узлы графа так, чтобы сначала шли узлы с заданными притоками (из $ V_Q $), 
а затем с заданными давлением (из $ V_P $) и разобьем векторы и матрицы на соответствующие
блоки:

$$ P = \left(\frac{P_{var}}{P_{fix}}\right) Q = \left(\frac{Q_{var}}{Q_{fix}}\right) A = \left( \frac{A_Q}{A_P} \right) A_S = \left( \frac{A_{SQ}}{A_{SP}} \right) A_F = \left( \frac{A_{FQ}}{A_{FP}} \right) \eqno(9) $$

Тогда уравнения (7) и (8) можно переписать 

$$ dX = (D_F A_{SQ}^T + D_L A_{FQ}^T)dP_{var} + (D_F A_{SP}^T + D_L A_{FP}^T)dP_{fix} \eqno(10) $$
$$ dQ_{fix} = A_Q (D_F A_{SQ}^T + D_L A_{FQ}^T)dP_{var} + A_Q ( D_F A_{SP}^T + D_L A_{FP}^T)dP_{fix} \eqno(11) $$
$$ dQ_{var} = A_P (D_F A_{SQ}^T + D_L A_{FQ}^T)dP_{var} + A_P ( D_F A_{SP}^T + D_L A_{FP}^T)dP_{fix} \eqno(12) $$

Матрица $ M = A_Q (D_F A_{SQ}^T + D_L A_{FQ}^T) $ связывает подвекторы с фиксированными переменными (заданными граничными условиями) с подвекторами свободных переменных.
Эта матрица также называется модифицированной матрицей Максвелла или $M$-матрицей.

Отметим еще несколько матриц
$$ M_{PP} = A_P ( D_F A_{SP}^T + D_L A_{FP}^T) \eqno(13) $$ 
$$ M_{PQ} = A_P ( D_F A_{SQ}^T + D_L A_{FQ}^T) \eqno(14) $$
$$ M_{QP} = A_Q ( D_F A_{SP}^T + D_L A_{FP}^T) \eqno(15) $$

Тогда (11) и (12) можно переписать используя (13), (14) и (15)
$$ dQ_{var} = M_{PQ} M^{-1} dQ_{fix} + ( M_{PP} - M_{PQ} M^{-1} M_{QP} )dP_{fix} \eqno(16) $$
$$ dP_{var} = M^{-1} dQ_{fix} - M^{-1} M_{QP} dP_{fix} \eqno(17) $$

Заметим также, что матрицы $$ M, M_{QP}, M_{PQ}, M_{PP} $$ -- функциональные матрицы, векторных аргументов $ P_0, Q_0 $