\chapter{Теоретическая часть}
\section{Постановка задачи}

Пусть G -- ориентированный граф с $ N_v $  узлами (образующими множество узлов $ E $) 
и  $ N_e $ ветвями (образующими множество ветвей $ E $). Расход по $ i $-й ветви связан 
с начальным и конечным давлениями  $ p_F^i $ и $ p_L^i $ замыкающим соотношением

$$ x_i = \phi \left ( p_F^i, p_L^i \right) \eqno (1) $$

Если $ A $ - матрица инцидентности графа $ G $
$ (a_{ij} = 1 $, если ребро j начинается в узле i, $ a_{ij} = -1 $, если ребро j заканчивается в узле i);

$ Q $ - вектор узловых 
притоков. Тогда уравнения Кирхгофа (уравнения балансов в узлах) записывается в виде

$$ AX = Q \eqno(2) $$

Используя матрицы $ A_F $ и $ A_L $, соответствующие выходящим и входящим ветвям ($ A = A_F + A_L $), 
вектор узловых давлений  $ P $ и вектор $ \Phi $ функций $ \phi_i $, уравнения (1) можно
записать в виде

$$ P_F = A_F^T P, P_L=-A_L^T P $$
$$ X = \Phi(P_F, P_L) \eqno(3) $$

Преположим что граничные условия заданы 

$$ P_{\gamma} = ( P_{i_1}, \dots, P_{i_k} ), Q_{\gamma} = ( Q_{i_{k + 1}}, \dots, Q_{i_{n}} ) \eqno(4) $$

Таким образом система уравнений

$$ X = \Phi (P_F, P_L), \\ \widetilde A X = Q \eqno (5)$$

Матрица $ \widetilde A $ - это матрица $ A $, только без последней строки, так как $ rank A = min(N_e, N_v) \leq N_v $.

При граничных условиях (4), система (5) имеет единственное решение. 

После решения системы (5), получаются векторы $ Q_0 $, $ P_0 $

Затем, предположим, что граничные условия (4) -- получили малые приращения соответственно 
$$ \widetilde P_{\gamma} = P_{\gamma} + \delta P_{\gamma},  \widetilde Q_{\gamma} = Q_{\gamma} + \delta Q_{\gamma} $$
Требуется оценить влияние изменений граничных условий на неграничные (незаданные) переменные.