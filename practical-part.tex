\chapter{Практическая часть}
\section{Пример}
Рассмотрим систему уравнений (5) и её решение $ P_0, Q_0 $
$$ A \Phi(A_F^T P_0, -A_L^T P_0) = Q_0 $$
А также перенумерованные векторы $ P_0 $ и $ Q_0 $ так, чтобы сначала шли узлы с заданными притоками, затем с давлениями

$$ P_0 = \left(\frac{P_0^{var}}{P_0^{fix}}\right), Q_0 = \left(\frac{Q_0^{var}}{Q_0^{fix}}\right) $$
и их "малые" изменения

$$ \delta P_0 = \left(\frac{ \delta P_0^{var}}{ \delta P_0^{fix}}\right), \delta Q_0 = \left(\frac{ \delta Q_0^{var}}{ \delta  Q_0^{fix}}\right) $$

В соотстветствие с описанным выше способом, получается:

$$ dQ_0^{var} = [ M_{PQ} M^{-1} ] ( A_F^T P_0, -A_L^T P_0  ) dQ_0^{fix} + $$ $$ [ M_{PP} - M_{PQ} M^{-1} M_{QP} ] ( A_F^T P_0, -A_L^T P_0 ) dP_0^{fix} $$

$$ dP_0^{var} = [M^{-1}](A_F^T P_0, -A_L^T P_0) dQ_0^{fix} - [M^{-1} M_{QP}](A_F^T P_0, -A_L^T P_0) dP_0^{fix} $$