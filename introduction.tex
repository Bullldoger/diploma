\chapter{Введение}

Гидравлические расчеты трубопроводных сетей, в частности систем газосбора и газораспределения,
требуются при решении многих технико-экономических проблем, связанных с их проектированием и эксплуатацией. 
Задачи гидравлического расчета решаются в рамках теории гидравлических цепей.
Одним из недостаточно исследованных вопросов этой теории является оценка и контроль откликов модели
трубопроводной системы на изменение исходных данных -- граничных условий -- задачи.
Эти вопросы относятся к сфере чувствительности модели.

\section{Чувствительность модели}
Вместе с моделью газопроводной сети рассматриваются и её начальные и/или граничные условия. Данные условия являются необходимыми для поиска решения, которое является единственным. 

\paragraph{Чувствительность модели} характеризуется способностью реагировать на изменения начальных
или граничных условий.

При расчете гидравлических цепей, условия (начальные или граничные) предполагаются неизменными, стационарными, однако, важно понимать, что в общем случае таковыми не являются, поэтому возникает множество вопросов связанных с тем, какова же будет реакция на изменения условий, какие узлы и дуги, и как отреагируют на эти изменения, или другими словами, ставится вопрос о чувствительности конкретной модели к варьированию начальных и граничных условий. 

\section{Анализ чувствительности}
Анализ различных аспектов чувствительности требует
моделирования этой системы и использования наиболее
адекватных моделей. Расчет гидравлических цепей является
сложной процедурой (в плане времени и вычислительных
ресурсов). В связи с этим целесообразно использование матриц чувствительности системы, которые связывают изменения входных параметров рассматриваемой системы с выходными. 
\paragraph{Матрица чувствительности} --  матрица якоби отображения пространства начальных и граничных условий в пространство откликов модели. 
